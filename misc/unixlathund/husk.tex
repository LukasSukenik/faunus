\documentclass[a4paper,10pt]{article}
\usepackage[a4paper,vmargin={2cm,2.2cm}, hmargin={1.5cm}]{geometry}
\usepackage{hyperref}
\hypersetup{pdftex,bookmarks=true,bookmarksopen=true,colorlinks=true}
\begin{document}
\renewcommand{\thefootnote}{\fnsymbol{footnote}}%A sequence of nine symbols (try it and see!)

\begin{center}
\Huge
\textbf{The Foxy $\mathcal{LATHUND}$\\ for MacOS X \& Linux\footnote[2]{To a lesser extent.}}\\
\normalsize
\vspace{0.5cm}
\emph{Compiled by Mikael Lund, Bj\"orn Persson, Bo J\"onsson\\
Last updated \today}
\end{center}

\tableofcontents

%%%%%%%%%%%%%%%%%%%%%%%%%%%%%%%%%%%%%%%%%%%%%%%%%%%%%%%%%%


\section{Graphics}
\subsection{Convert vector graphics to bitmap with defined size and resolution}
This is useful for Word documents where correct graphic dimensions/resolution should be maintained \emph{before} importing. A hassle, but here's one way of doing it:
\begin{enumerate}
\item Open vector graphics in Preview.app
\item Create a new paper size in the desired format.
\item Print to PDF, selecting the new paper format.
\item Open new PDF in Gimp. You will be prompted for a resolution (DPI).
\item Save bitmap file to png, tiff etc.
\end{enumerate}
This could likely be done easier using ImageMagick's \verb"convert" command.

\subsection{Making vector illustrations}
Vector illustrations can be made in a number of different programs such as
\verb+Pages+, \verb+Keynote+, \verb+Powerpoint+ etc. To export them to a PDF file do as follows:
\begin{enumerate}
\item Create illustration and select the objects you wish to export.
\item Select \verb+Edit->Copy+
\item Launch Preview.app and select \verb+File->New From Clipboard+.
\item Export to PDF or any other format.
\end{enumerate}

\subsection{Conversion commands etc.}
\begin{table}[h!]
\center
\begin{tabular}{ll}\hline\hline
Action              & Command \\\hline
Crop blank PDF borders & \verb"pdfcrop file.pdf" \\
Convert PDF/EPS to 300 dpi tiff & \verb"convert -compress lzw -density 300x300 in.pdf out.tif" \\
Convert Xmgrace plot to pdf     & \verb"xmgrace -hardcopy -hdevice PDF file.agr"\\
Convert to monochrome           & \verb"convert -threshold 38% -despeckle out.jpg -monochrome in.jpg"\\
Print syntax-highlighted source code & \verb"enscript --media=a4 -E sourcefile"\\
Convert DVI to PS with \textbf{pretty} text & \verb"dvips -Ppdf t.dvi -o t.ps"\\
Convert PS to PDF                           & \verb"pstopdf t.ps"\\
\hline
\end{tabular}\end{table}

%%%%%%%%%%%%%%%%%%%%%%%%%%%%%%%%%%%%%%%%%%%%%%%%%%%%%%%%%%


\section{Networking}
\subsection{Access scientific journals from home}
Establish an encrypted SSH-tunnel to any computer situated on campus:
\begin{enumerate}
\item For example: \verb+ssh -D 9999 mimer.teokem.lu.se+
\item In \verb+System Preferences->Network->Advanced->Proxies+ set SOCKS proxy to \verb+localhost+, port 9999.
\end{enumerate}
All your web browsing will now appear as if you are situated inside campus. If you have a slow connection, you may want to add ``-C'' in the ssh command to enable compression.  When you close the ssh connection you must again remove the proxy setting in the network preferences to return to ``normal'' browsing.

\subsection{Automatic SSH login}
This will allow login/scp to ssh servers without entering your password:
\begin{enumerate}
\item Protect your local .ssh dir: \verb"chmod 700 ~/.ssh"

\item If you haven't got one, create a public ssh key.
Look at \verb"~/.ssh". If you see a file named id\_dsa.pub then you already have a public key. If not, create one with: \verb"ssh-keygen -t dsa"

\item Transfer your public key to the remote server:\\

\verb"$ cat ~/.ssh/id_dsa.pub | ssh user@server 'cat >> ~/.ssh/authorized_keys'"\\

(In this step you may be able to use \verb"ssh-copy-id user@server" instead.)
\end{enumerate}

\subsection{Mount remote disks from SSH-servers}
For this \verb+sshfs+ needs to be installed -- for example using MacPorts or MacFUSE. In this example we will mount /disk/gloabal/mikael from the host milleotto,\\

\verb+$ sshfs milleotto.lunarc.lu.se:/disk/global/mikael mntdir+\\

\noindent The remote disk will appear in ``mntdir'' and you'll have full read and write access. To unmount, simply do \verb+umount mntdir+. As usual ``-C'' will enable compression. \textbf{Warning:} if the mntdir resides in your home directory, the remote disk may be included in the hourly/daily backup!

\subsection{Printing}
\subsubsection{Automatic CUPS detection}
CUPS printer detection has been disable in some versions of MacOS (10.5, higher?). Use the following command to re-enable automatic discovery:\\

\verb+cupsctl BrowseRemoteProtocols=cups+

\subsubsection{Connect to a specific CUPS server}
If auto-detection fails, try adding ``ServerName SERVER'' to /etc/cups/client.conf. Note that this usually causes printing to hang if the computer is disconnected from the network and, hence not recommended for laptops.

%%%%%%%%%%%%%%%%%%%%%%%%%%%%%%%%%%%%%%%%%%%%%%%%%%%%%%%%%%

\section{Software}
\subsection{Port Managers}
$\mathcal{A}$re handy interfaces to automatically download and install a large number of open source software in a self-consistent manner. Usually they require that Xcode -- Apple's developer and compiler tools -- are installed.
\subsubsection{Macports}
\begin{table}[h!]
\center
\begin{tabular}{ll}\hline\hline
Command              & Action \\\hline
\verb"port search keyword" & search for a package\\
\verb+port variants packagename+ & show \emph{variants} of a package.\\
\verb"sudo port install packagename" & install a package.\\
\verb"sudo port install emacs +x11" & install emacs variant with X11 support.\\
\verb"sudo port -d selfupdate" & update entire package system (do this only occasionally)\\\hline
\end{tabular}\end{table}
%
Useful packages are \verb+grace gnuplot povray sshfs vim aspell wget gawk tetex findutils+

\subsubsection{Fink}
\begin{table}[h!]
\center
\begin{tabular}{ll}\hline\hline
Action              & Command \\\hline
Search for package  & \verb"apt-cache search package"\\
Install package     & \verb"sudo apt-get install package"\\
Update packages     &  \verb"sudo apt-get update|upgrade"\\
Install source fink package & \verb"sudo fink install package"\\\hline
\end{tabular}\end{table}

\subsection{Visual Molecular Dynamics}
Or just VMD is a very impressive program for molecular visualization/analysis and although the name refers to MD, it works nicely with MC. A speciel feature of the mac port is that the command line tool is buried. Here are two ways to surface it:
\begin{itemize}
\item[\$] \verb+ alias vmd='/Applications/VMD\ 1.8.6.app/Contents/vmd/vmd_MACOSXX86'+
\item[\$] \verb+ sudo ln -s /Applications/VMD\ 1.8.6.app/Contents/vmd/vmd_MACOSXX86 /usr/local/bin/vmd+\end{itemize}

\subsection{Faunus}
To fetch the newest revision from the subversion repository, do\\

\verb+$ svn checkout http://faunus.svn.sourceforge.net/svnroot/faunus/trunk faunus+\\

\noindent Make sure to specify the compiler in the Makefile -- in particular parallel execution works, as of now, much better on Intels C++ compiler than with g++.

\subsection{Grace}
Some tips:
\begin{itemize}
\item White borders in graphs can be removed in \LaTeX{} by including graphs with\\

\verb+\includegraphics[clip]+
\end{itemize}

\subsection{ACE/gr or ``Xmgr''}
Yikes! This is a hassle to compile on recent versions of MacOS since development has been discontinued for a decade. PowerPC versions (compiled via Fink) run on Intel macs via transparent rosetta emulation. I have prepared a MacOS X installer that should work on MacOS X 10.4--10.5, PowerPC or Intel:\\

\url{http://idisk.mac.com/mlund/Public/Software/Xmgr.pkg.zip}

%%%%%%%%%%%%%%%%%%%%%%%%%%%%%%%%%%%%%%%%%%%%%%%%%%%%%%%%%%
\section{\LaTeX{} typesetting}
\subsection{Getting started}
\begin{enumerate}
\item Create a \LaTeX{} file, ``test.tex'', for example from the following template that shows some of the basic commands:
\begin{verbatim}
\documentclass[a4paper,12pt]{article}
\usepackage[a4paper,vmargin={2cm,2.2cm}, hmargin={1.5cm}]{geometry}
\usepackage{hyperref} % Enables clickable links for on screen reading
\usepackage{graphicx} % Enables graphics import
\hypersetup{pdftex,bookmarks=true,bookmarksopen=true,colorlinks=true}
\begin{document}
\section{Some section}
\subsection{Some subsection}
This is Equation~\ref{coulomb}:
\begin{equation}
\beta u = \frac{l_b z_iz_j}{r_{ij}} \cdot \exp({-\kappa r_{ij}})
\label{coulomb}
\end{equation}
...and some inline math, $\sqrt{x}$, like this.

Also web-links can be useful for on screen reading:
\href{http://www.bmj.com/cgi/content/full/319/7225/1596}{[Exciting experiments]}
\end{document}
\end{verbatim}
\item Compile with: \verb"pdflatex test.tex" to automatically produce a PDF file with hyperlinks and bookmarks.
\end{enumerate}
Note that using pdf\LaTeX{} any graphics inserted with \verb"\includegraphics" must be in the format pdf, png, jpg, mps, or tif. EPS files \emph{cannot} be used with pdf\LaTeX{} and must be converted first.\footnote{This can be done automatically using the auto-pst-pdf package -- see Section~\ref{sec:psfrag}} For example:\\

\verb"$ epstopdf file.eps"

\subsection{Attaching papers to a document}
Insert \verb"\usepackage{pdfpages}" in the header and attach pdf files -- anywhere -- with\\

\verb"\includepdf[pages=-]{file.pdf} % For all pages in the file"\\

\verb"\includepdf{file.pdf} % For the first page only"\\

\noindent Fancy stuff like scaling and advanced page layouts are possible; see the manual for \verb"pdfpages".

\subsection{Attaching files to a PDF document}
Files (programs, images etc.) can be attached to PDF documents. In pdf\LaTeX{} this is done via the \verb"attachfile" package that allows file insertion with \verb"\attachfile{filename}".

\subsection{\label{sec:psfrag}Using \LaTeX{} commands in EPS figures}
Text labels in eps files generated with Xmgr, gnuplot etc. can be replaced with \LaTeX{} commands for advanced math symbols and equations.
\begin{enumerate}
\item Do like this:
\begin{verbatim}
\documentclass{article}
\usepackage[on]{auto-pst-pdf}
\begin{document}

\begin{figure}
\psfrag{MYAXIS}{$\sqrt{x+y}$} % Replace "MYAXIS" in fig with latex math.
\includegraphics{epsfile.eps}
\end{figure}

\end{document}
\end{verbatim}
\item latex/pdflatex with the argument \verb"-shell-escape" to produce either a dvi or pdf file. Note that the \verb"auto-pst-pdf" package automatically converts EPS files to PDF when using pdf\LaTeX{}.
\end{enumerate}

%%%%%%%%%%%%%%%%%%%%%%%%%%%%%%%%%%%%%%%%%%%%%%%%%%%%%%%%%%
%\section{General UNIX command}
%%%%%%%%%%%%%%%%%%%%%%%%%%%%%%%%%%%%%%%%%%%%%%%%%%%%%%%%%%

\newpage
\section{Table of useful commands}
\begin{table}[h]
\begin{small}
\begin{tabular}{ll}\hline\hline
Action              & Command \\\hline
\textbf{Processes Handling}\\
Quick and dirty suicide!       & \verb"kill -9 -1"\\
Pause/resume job     & \verb"kill -s STOP|CONT pid" \\
Terminate job by name     & \verb"killall jobname" \\
Launch logout-safe job  & \verb"nohup program &" \\
Be a gentleman, run in low priority & \verb"nice program" (use when exploiting a colleagues computer)\\
\\
\textbf{Files and search}\\
Disk information   & \verb"df -h"\\
Space used in current directory & \verb"du -h" (or \verb"-csh" for a summary)\\
Find file           & \verb'find * -iname "file*"'\\
Find large files    & \verb'find * -size +1000k printf "%p %k \n"'\\
Find files and contents (mac) & \verb'mdfind text...'\\
``Double-click'' file (mac)     & \verb"open file"\\
Mount disk image (mac)    & \verb"hdid image.dmg"\\
\\
\textbf{Archives}\\
Create zip archive  & \verb"zip -r zipfile.zip file(s)"\\
Test zip archive    & \verb"unzip -t zipfile.zip"\\
Unpack zip archive  & \verb"unzip zipfile.zip [file(s)]"\\
Create tar archive  & \verb"tar -czf archive.tar.gz directory"\\
List tar archive    & \verb"tar -tzf archive.tar.gz"\\
Unpack tar archive  & \verb"tar -xzf archive.tar.gz"\\
Send files via tar and SSH & \verb+tar czf - file(s) | ssh host "cat > archive.tar.gz"+\\
                    & (For \verb".tar.bz2" files, replace \verb"z" with \verb"j")\\
\\
System information  & \verb"cat /proc/cpuinfo" (linux) \\
                    & \verb"system_profiler" (mac) \\
\\
X11 focus           & \verb"defaults write com.apple.X11 wm_ffm -bool true"\\
Terminal.all focus  & \verb"defaults write com.apple.Terminal FocusFollowsMouse -string YES"\\
\\
Get svn repository   & \verb"svn checkout URL destdir"\\
Update to latest repository & \verb"svn update"\\
Review changes       & \verb"svn status" (go to top level dir to see all)\\
Compare with latest repository & \verb"svn status -u" (if astrisks, update)\\
Compare difference's after modification& \verb"svn diff"\\
Submit svn changes   & \verb"svn commit"\\
Add file or directory& \verb"svn add file"\\
Show log             & \verb"svn log"\\
Regret and undo modification & \verb"svn revert foo"\\
Remove conflict files        & \verb"svn resolved"(not need if using revert)\\
\hline
\end{tabular}
\end{small}
\end{table}

%%%%%%%%%%%%%%%%%%%%%%%%%%%%%%%%%%%%%%%%%%%%%%%%%%%%%%%%%%

\section{Specific to Theoretical Chemistry, Lund University}
\subsection{Home directories (outdated?)}
User home directories are served via NFS by \verb'garm.teokem.lu.se'. Some
versions of Linux are not entirely compatible with MacOS which can cause
problems with file locking. This is used in \verb+Mail.app+, \verb+Address Book.app+
etc. and these programs will NOT work when home is on garm (as of
writing, Dec. 2006). To circumvent this \verb+~/Library+ can be placed on a
local disk and a symlink placed in the home directory (ln -s).
To setup MacOS for NFS to garm I suggest using the free program
\verb+NFS manager+ by Bresink.
There you will also find a nice manual on integrating MacOS in NFS
and NIS environments. When setup, the mount point can be checked in
\verb+NetInfo Manager+ and should have the following options: \verb+-s,-P,-b+.
A restart is most likely required for the changes to take effect.


\end{document}

